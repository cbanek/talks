% super simple template for automated 2018 ADASS manuscript generation from the registration entry
% place this file in your ADASS2018_author_template directory
%
% Only few comments here, see the ADASS_template.tex for a more fully commented version, and
% ManuscriptInstructions.pdf if you need more background, and if you even need more, APS's own
% manual2010.pdf has it all!

% Version 16-aug-2019 (Erik Deul)

\documentclass[11pt,twoside]{article}
\usepackage{asp2014}

\aspSuppressVolSlug
\resetcounters

\bibliographystyle{asp2014}

\markboth{Banek}{(I6.1) Why is the LSST Science Platform built on Kubernetes?}
% remove/add as you need

\begin{document}

\title{(I6.1) Why is the LSST Science Platform built on Kubernetes?}


\author{Christine~Banek$^1$}
\affil{$^1$AURA/LSST, Tucson, AZ, USA; \email{cbanek@lsst.org}}
% remove/add as you need

% remove/add authors as you need
\paperauthor{Christine~Banek}{cbanek@lsst.org}{0000-0002-4337-4956}{AURA}{LSST}{Tucson}{AZ}{85716}{USA}
% remove/add as you need

% leave these next few aindex lines commented for the editors to enable them. Use Aindex.py to generate them for yourself.
% first presenting author should be the first entry for bold-facing the author index page-reference
%\aindex{Banek,~C.}
%\aindex{Author2,~S.}
% remove/add as you need

% leave the ssindex lines commented for the editors to enable them, use Index.py to suggest yours
%\ssindex{FOOBAR!conference!ADASS 2018}
%\ssindex{FOOBAR!organisations!ASP}

% leave the ooindex lines commented for the editors to enable them, use ascl.py to suggest yours
%\ooindex{FOOBAR, ascl:1101.010}

\newcommand{\code}[1]{\texttt{#1}}

\begin{abstract}
LSST has chosen Kubernetes as the platform for deploying and
operating the LSST Science Platform.  We first present the
background reasoning behind this decision, including both
instrument agnostic as well as LSST specific requirements.
We then discuss the basic principals of Kubernetes, and how
they are used as the basis for the LSST Science Platform.
We conclude with an example of how an outside group may
use the publicly available resources to deploy their own
instance of the LSST Science Platform, and customize it
to their needs.
\end{abstract}

\section{Introduction}

The Large Synoptic Survey Telescope (LSST) will produce a huge amount
of data over its 10 year survey. A dataset this large is not easily stored,
copied, or manipulated.  To store the data and process it, hundreds of
machines will be required.

The LSST Science Platform (LSP) provides next to data processing and real
time interaction with the data.  By allowing scientists to run their own
custom python code in Jupyter notebooks in the LSP, we place them near the data,
which allows for much higher bandwidth access to the data than between
computers on the internet, as well as access to more processing
resources for doing large scale analysis over the dataset.

\subsection{Without Reproducbility, No One Knows What You Found}

The processes and tools used to manage these systems must also scale and
support these needs.  At the core of these needs is the ability to
reproducibly run software on machines, which has always seemed easier
in theory than in practice.

Over the 10 year survey, machines will naturally fail, or be replaced.
Software will be updated and reinstalled.  The system
must be able to work assuming that some of its nodes will be offline at
any given time, because some inevitably will be with such a large number
of machines.  As the system also needs to support user demand, the system
must also be able to appropriately scale to load.

These practical demands require an automated and reliable process for
deploying new systems, or the whole system from scratch, in a reproducible
way.  If it was not automated, then too much manual effort would be involved,
taking time and resources.  Even well intentioned manual work by knowledgeble
professionals is never perfect, possibly leaving unnoticed mistakes that could
affect performance or even scientific results.

While your project may not suffer from the scale of LSST, I would argue it is
even more important to have reproducible software.  By having fewer machines,
a larger percentage of your computing resources will be unavailable if struck by
hardware or software failures, possibly to the point of degraded performance
or unavailability of the entire system.  If, however, you are lucky and your
few machines take many years to fail, it may be found too late that you cannot
easily reproduce the configuration due to software being deprecated or deleted.
Longer lived machines also have greater risk of being manually tampered with by
well meaning operators and developers attempting to quickly fix or diagnose
solutions.

Results are only as good as your ability to reproduce them, large or small.
Reproducing scientific results on a non-reproducible software platform of any size
is fraught with risks.  While software will never be perfect, by tightly controlling
the conditions in which it runs, results should at least be easily reproducible.
If it's not, no one knows if you've found a software bug or the next Nobel prize.

\subsection{The Quest for Reproducible Software}

Reproducing a working software environment is not a trivial thing, due
mostly to the number of interconnected pieces involved.  There's the software
you are trying to run (the applications themselves), libraries and runtimes that
those applications require, the operating system, and other applications running with
or by the application on the same machine. More often than not, one application
will also call another over the network, which requires knowledge of where that
application is hosted (DNS name, URL) as well.

Most software is designed to run under very controlled conditions, because
it makes software easier to write, test, and verify.  Having to support every version of every
operating system is a lot of work that doesn't make sense for many software
projects, especially when the software won't need to run under those constraints
in production.

Enter virtualization and the virtual machine (VM).  Now on your development machine running
Windows, you can run a linux server with a particular configuration that behaves just
like production.  Disk images and unattended installs for VMs provided a standard way of
creating machines with a cookie cutter configuration.

Running VMs in production was still hard.  VMs can introduce a fair amount of performance
overhead with heavy processing or disk I/O.  Each VM also has its own copy of the operating
system, libraries, and utilities, increasing the overhead of placing many VMs on the same
physical host.  Using VMs, it can be very costly to provide complete isolation of application
instances, but it was possible.  Many hybrid approaches with different flavors of machines running
a particular set of services was very popular and successful, but this lacked flexibility
in the ways a system could scale.

\subsection{Containerization}

Containers are the current industry solution to this problem.  Containers are a way of packaging
the application, its dependencies, and configuration as one unit.  Many containers can be
run on one host sharing the same kernel.  This reduces the overhead of running many containers
versus running many VMs, while still providing much of the isolation of a VM.

\footnote{Containers use various features in kernels such as namespacing and cgroups
to share the same kernel and have isolated filesystems and process spaces.}

Since containers have everything they need already installed, and without the overhead of
starting an operating system, containers can start very quickly, typically within seconds.
This makes them not only reliable but quick to start on a new machine.

Containers can also be published to the internet.  hub.docker.com allows for users to
create accounts and publish their own container images, which are basically a compressed
tar-file of the root filesystem.  This allows for anyone to download your container and
start it on their machine.  When published, containers generally have a repository name,
an image name, and a tag that contains some version information.  This is typically
written as \code{repo\_name/image\_name:tag}.

So, finally, we have a reliable method to reproduce the installation of our software,
at least on one machine.

\subsection{Kubernetes}

While containers are very useful, one container doesn't know about the other containers
on the system, and doesn't know anything about what might be running on other systems.
In order to build complex systems from these containers, we need an orchestrator that
can stitch together containers across multiple machines into one logical system.  Enter
Kubernetes.

Kubernetes, or k8s for short, is an open-source system built originally by Google to handle this exact
problem.  Kubernetes allows for the coordination and orchestration of containers across
a cluster of machines.

\section{Kubernetes: Core Concepts}

K8s can be complex for newcomers, and there are many great tutorials out there
for K8s.  While this isn't intended as an introduction to K8s, we will
introduce a few of the basic concepts to provide a common vocabulary for
those who are not familiar.

K8s is a declarative system - instead of running
a sequence of commands, you inform k8s what you want the state of the system to be,
and it will take the required actions to get to that declared state, as well as maintain
that state over time.  K8s generally gets the declared state through a number of YAML documents
that represent Kubernetes resources that are applied using the command line utility \code{kubectl}

Let's learn about the different Kubernetes resource types through a 'hello world' example.
\footnote{Here is a full tutorial that goes command by command through this process: https://kubernetes.io/docs/tutorials/hello-minikube/\#create-a-deployment}
First, let's say you have a container named \code{myorg/hello\_world:latest}, which is
the latest version of the \code{hello\_world} container, and you want to run this on kubernetes.

The Kubernetes way of thinking about this is that you want to create a \code{deployment} of this
container.
\footnote{Here is a good hands on tutorial with command lines: https://kubernetes.io/docs/tutorials/hello-minikube/\#create-a-deployment}
\footnote{Technically, deployments act on pods, which are groups of containers that should be scheduled next to each other.  But in this example, it is a pod with one container.}
A \code{deployment} is a k8s resource that instructs Kubernetes to check the health of your container,
and take actions to ensure it is kept running.  If the container exits, doesn't respond
to a health check, or the machine hosting it becomes unavailable, that container is then restarted
or moved to a new machine.

Deployments also allow a number of replicas to be set.  To run more copies of your application to
scale it up, you can use the \code{kubectl} tool to increase or decrease the number of replicas for
a deployment.  If the number of running copies falls below the number of desired replicas, new
containers are created.

Kubernetes doesn't need to be told where to run the container.  Based on different heuristics,
it will pick a machine to start the container on.  However, labels and restrictions can be placed
on nodes and applications to ensure containers are run on particular groups of machines.
\footnote{https://kubernetes.io/docs/concepts/overview/working-with-objects/labels/}

At this point, you have a running container, but you don't know where it is running, so therefore
there is no way to access it.  Moreover, since a container might be moved or restarted somewhere
else, you need to know how to contact the currently running container, wherever it is.  By creating
a \code{service} resource, you can present a group of containers as an internal DNS name for your
service.  This DNS name is taken from the \code{service} name, and will loadbalance
requests amongst all the running containers using round-robin DNS.

By default, containers running in k8s are connected to a private internal network shared by all
the k8s nodes, and is not accessible from the internet.  To allow the container to be reached
from the internet, you will need an \code{ingress} resource.
\footnote{The other way is to set the type of the service to LoadBalancer, which on many
cloud providers allocates an external IP address, but this configuration is not available
everywhere.}

An \code{ingress} resource allows a k8s service to be contacted from an external network,
such as the internet.  \code{ingress} resources have a mapping of hostnames and HTTP(s) paths
to backend k8s \code{service} instances.  These requests are routed through an ingress
controller, such as the \code{nginx-ingress} controller, which we will use later on.
Each ingress rule can have individualized k8s annotations to configure timeouts,
or use advanced features such as URL rewriting.  By using \code{ingress} resources you can
avoid the pattern of having a different nginx or load balancer in front of each service.
It is also easy to host multiple web services or applications behind one public DNS name,
routing them by the URL.

Some kind of permenant storage is typically required for applications, and since k8s may move containers
between machines, being able to find and utilize the same storage is an important
consideration.  K8s provides a way to allocate and attach storage to your container.
The \code{persistent volume} or \code{PV} tells k8s what type of storage to use, and how
much space to allocate.  Each \code{PV} then be mounted by creating a \code{persistent
volume claim}, and using that claim in the \code{deployment} YAML.  In this way, your
container can mount a \code{PVC} on a specific path, making it look just like another
filesystem.  However, unlike the container filesystem, the \code{persistent volume}
storage will be kept after the lifecycle of the container, allowing it to hold persistent
state across container restarts.

K8s provides a few ways of configuring services and managing configuration files.  The
simplest is the container environment.  In the \code{deployment} YAML, each container
is defined, including the UNIX environment variables to set for that container.  In this
way, a container when running can look at the variables injected into its local environment
for configuration.  Due to the ubiquitous use of environment variables for programs, intended
for containers or not, this can be a very simple way to configure your container.

Another type of configuration is the \code{configmap}.  K8s \code{configmap}s are typically
groups of files each with a name.  Like mounting a \code{persistent volume}, a \code{configmap}
can be mounted into a container, presenting the files in the configmap as normal files to be
read by any application.

One last type of configuration to mention is the k8s \code{secret}.  K8s \code{secret}s can
be used to hold sensitive configuration files of any type, passwords, or certificates.  Special
care is taken to not expose the values of these secrets, which are typically shown for
environment variables and configmaps.  By using \code{secret}s it is easier to tell which
parts of the configuration are secret (and likely instance specific) and which are general
configuration variables.  \code{secret}s can be mounted like a volume into the containers
that they are used, or imported into the environment variables.

Finally, k8s also has the concept of different \code{namespaces} to provide isolation between
different applications running on the system.  \code{namespaces} not only provide a way to
compartmentalize applications, but also to control them with roles and quotas.

\footnote{Using roles and quotas is an important feature, but outside the scope of this paper.  Here are some helpful links:
https://kubernetes.io/docs/concepts/overview/working-with-objects/namespaces/
https://kubernetes.io/docs/concepts/policy/resource-quotas/ https://kubernetes.io/docs/reference/access-authn-authz/rbac/}

\section{Helm}

K8s can be intimidating, and each application can have many parts.  For people who just
want to install the software and run it, rather than develop it, looking at the raw
k8s YAMLs might be too confusing or create a high barrier to entry.

Over the lifetime of k8s, many groups have tried to solve this problem, and one of these
solutions is Helm.\footnote{https://helm.sh}  Helm is a package manager for k8s, allowing
for installation of all the parts of an application in one command, similar to apt-get
or yum on Linux.  Helm also allows for upgrades and rollbacks of software versions.
Most importantly, Helm allows for customization of the installed software, which is
especially important to configure the software correctly for your k8s environment.

Helm's packages are called charts.  Each chart is a tar archive containing a set of
k8s YAML files to apply to the cluster, but these YAML files are first run through
a templating engine.  This allows for a chart developer to take user provided values
and place them in the right parts of the final YAML files it will send to k8s.
The values to apply to the template are provided via a user specific YAML document.
This allows for the user to build a simpler YAML document with just the configuration
values they want to apply, without having to have deep k8s knowledge.  In this way,
users can now install software without needing to know how to develop that software,
which lowers the barrier to entry.

Helm is one of the leading software packaging schemes for k8s, and many charts
already exist for commonly used open-source software.\footnote{https://github.com/helm/charts
is the repository for official Helm charts}

\section{Using Helm to Installing the LSP}

To help everyone be able to install the LSP, there is now a repository of helm
charts available at https://github.com/lsst-sqre/charts.  Each of these components
of the LSP has its own chart that anyone can install:

\begin{itemize}

\item landing-page: A simple HTML landing page for the LSP with links to help users
discover the different components of the LSP.  This also provides a simple message of
the day to users, allowing operators and administrators to present messages about
system upgrades or downtime.

\item fileserver: A simple NFSv4 fileserver that provides a shared storage for
home directories an data.  This is mounted by the Nublado JupyterLab containers,
and allows for persistent storage between logins, as well as a place to centrally
store data to process.  For those with an existing fileserver in the environment,
this won't be used, but instead Nublado can be configured to point to your existing
fileserver.

\item nublado: LSST's JupyterHub environment.  Nublado allows for users to spawn
JupyterLab containers in k8s.  Users can run Jupyter notebooks and/or do interactive
data analysis and investigation next to the data.  This chart provides a number of
features over stock Jupyter, including a page to allow users to pick the version of
their Jupyter environment, as well as more tightly integrated namespace and quota
features.

\item firefly: The LSP provides firefly as one of the ways users can interact with
data using their browser.  Firefly is a portal that allows for quick preliminary
investigation using the browser, as well as being integrated into the Jupyter
environment for interactive analysis with notebooks.

\item cadc-tap: The CADC TAP chart installs an IVOA TAP (Table Access Protocol)
server into k8s, allowing for users to query catalogs through nublado, firefly,
and externally.  This chart also sets up simple databases in k8s to provide
TAP\_SCHEMA and a sample dataset for the TAP service to serve.

\end{itemize}

As we add more services and features to the LSP, they will also get Helm charts.

Convenience scripts with a working configuration for Google are provided here:
https://github.com/lsst-sqre/lsp-deploy

Given that the instructions may change over time, I won't go through the exact
instructions here, but explain what is happening.  The authorative instructions
are in the repository along side the code, so they can be updated together.

There are a few stages to installing the LSP.

\subsection{Creating a Kubernetes Cluster}

Either you have an existing k8s cluster, or you will create one.  It is very
easy to create a k8s cluster on Google using Google Kubernetes Engine (GKE).
While this can be done on the command line, it can be done with little effort
using the Google Cloud Console.

\footnote{https://cloud.google.com/kubernetes-engine/docs/how-to/creating-a-cluster
provides a good walkthrough of creating a cluster using both the command line
and the web portal.}

In general, you want your k8s cluster to contain at least 3 nodes, each node
having at least 2 if not 4 cores.  Also be sure to alter the host disk from
the default 100GB to a higher number to store all the container images, about
200GB should be safe.

Once you've created your cluster, you can connect it to your local kubectl
command by clicking the 'Connect' button, and running the command line provided.

\subsection{Setting up Helm}

Helm has a local component and a k8s component that both need to be installed.
The \code{install\_tiller.sh} script provides a simple wrapper to set up Helm.
By default, this isn't particularly secure, so you should look into the Helm
documentation to create a more secure deployment, but it will allow us to
continue and install the various LSP components.  You can tell if helm is
working properly by running \code{helm ls}, and returning that nothing
is installed, but no error is generated.

\subsection{Configuring nginx-ingress}

The next step is installing and configuring nginx-ingress using the Helm chart.
These steps are provided in \code{install\_ingress.sh}.  This script creates
a k8s secret holding your SSL certificate for the cluster, installing the Helm
chart, and configuring nginx to use the created secret to find the certificate.

If you don't already have a current SSL certificate, you can create one for
free using letsencrypt.org.\footnote{https://letsencrypt.org/getting-started/}

Now there is a manual step of setting the DNS record for your new cluster.
Run the \code{public\_ip.sh} script to retrieve the external IP address for your
cluster, and set it to a DNS name that works with the SSL certificate provided.

At this point, if everything went well, you should be able to go to your
DNS name over https, and get a 404 that says \code{default backend}.

\subsection{Installing LSP Charts with Helm}

Now let's examine the \code{install\_lsp.sh} script.  This script adds the LSP
chart repository to Helm's search path, and uses \code{helm install} to install
the charts.  After you run this script, you should be able to run \code{helm ls}
and see that multiple charts are now installed, and their versions.  Now if you
return to your DNS name in the browser, you should be presented with the LSST
LSP Landing Page.  This provides links to the Nublado notebook environment and
Firefly.

Depending on the size of your cluster, it may take a few moments for all the
services to start and become ready.  You can check the status of different
k8s resources by running \code{kubectl get all -A} which will show all of
the k8s resources and their status.

\section{Customizing the Landing Page}

Now that you have a running LSP, it is easy to inspect it, use it, and see
how it works in practice.  One of the first things you might want to do is
customize the landing page.  After all, it would be confusing to present
yourself as the LSST Science Platform if you are not.  Here we will go
through a few different types of customizations as examples of how you
could customize the LSP for your needs.

\subsection{Changing the MOTD}

One simple thing you will want to do is change the message of the day,
which is presented in a box on the landing page.  After all, it would be
confusing to present messages that aren't intended for your users.
The message of the day comes from a markdown document whose URL is passed
into the landing page container as an environment variable.
We will use Helm to allow us to customize that parameter
to point to a new markdown document.  Create a file named langing-page-values.yaml
with the following contents.

\fbox{
motd\_url: <URL of your MOTD markdown document>
}

Now you can call helm install and add the argument --values, and pass the path to
the newly created file.  This will take the URL provided in the landing-page-values.yaml,
and apply it to the chart.  Now when you go to the landing page, you should see
your new MOTD message.

\subsection{Changing the Container Image}

Instead of just changing the MOTD, if you want to use a different HTML landing page,
the easiest way is to create a new container with the web page you want.  You can
see how the docker container for the landing page is created by looking at the
Dockerfile in the landing page repository.\footnote{https://github.com/lsst-dm/lsp-landing-page/blob/master/Dockerfile}

It is possible to either make a GitHub fork of this git repository or create
a new docker image using your own assets.  For the LSP landing page, all the assets
are in the public\_html directory of the git repository, and it is added to a directory
that nginx is ready to serve it from.

Once you make a new container and push it to hub.docker.com, you can create a
file named landing-page-values.yaml to point to that image.  If your new image is
\code{newtelescope/landing-page:latest} your landing-page-values.yaml would look like:

\fbox{
image:
  repository: newtelescope/landing-page
  tag: latest
}

\subsection{Fixing Bugs and Suggesting Chart Changes}

Bugs are the nature of software development, and Helm charts are no different.  If you
find a bug, please make an issue on the chart repo \footnote{https://github.com/lsst-sqre/charts}
on GitHub.  This will allow others to see there may be an issue, as well as notify the
developers that something may need a fix.

If you already know what the required fix is, please submit a pull request (PR)
on the repository with your changes.

\footnote{https://help.github.com/en/articles/creating-a-pull-request}

If you have new features you wish to add to an existing chart, such as new ways of overriding
particular deployment values, you can also submit a PR containing that information.  It helps
if the changes don't remove or alter other functionality, but sometimes this is required.

\section{Publishing Your Astronomy Software with Helm}

The Helm charts listed in section 4 are not intended to be a complete or authoritative
list, but a starting place.  I hope that after seeing the usefulness of being able
to install another institutions science platform, your institution will also want to
Kubernetes and Helm to share software with the rest of the astronomy community.

The Helm repository framework allows for anyone to pull charts from multiple different
repositories without having to have a centralized store, gate-keeper, or process.  It is
simple to set up your own repository on GitHub to store your own set of helm charts,
and proceed with whatever software development process or timeline your institution
recommends or requires.\footnote{https://helm.sh/docs/chart\_repository/ is an
excellent resource for using existing services such as GitHub Pages and others
to host your helm charts for free.}

When possible, it is better to patch or add features to an existing chart rather
than create a new chart that deploys the same software.  Having many charts that
accomplish the same thing can add confusion and maintenance overhead.

If you are creating a new chart for a new software service, there are many great
blog posts and helpful documentation on the Helm website.\footnote{https://docs.bitnami.com/kubernetes/how-to/create-your-first-helm-chart/
is a very thorough walkthrough of an example}

\section{Conclusion}

Astronomy as a field will be ever more reliant on software to produce its results.
Being able to reproducibly develop, deploy, configure, and utilize software are
essential first steps to relying on that software to produce novel scientific
results that can be reproduced, verified, and trusted.

As astronomy software becomes more complex and intertwined with other big
data science fields, such as computer science, biology, and physics, it will
become necessary to share software between these groups.  Writing software that
is of high quality and scales to these problems requires specialized talent,
and a lot of other resources.  It makes logical, scientific, and financial
sense that disparate fields work together to build, test, and verify software
together, instead of each institution developing their own specialized software.

In the era of big data and multi-messenger astronomy, it is practically required
for institutions that mirror an instrument's data to also provide local mirrors
of software services and tools to operate on that data.  Given the specialized
nature of each instrument, it will be increasingly important for environments
to allow astronomers to work with multiple datasets in the same datacenter
simultaneously.  These multi-instrument datacenters will require a standard
way of installing and updating software provided by the instrument creators
with limited assistance. Kubernetes and Helm provide one way of accomplishing
this goal using already standard software industry practices and tooling.

By bringing in tools and practices from the software industry, we can also
leverage the open source community (as well as companies who participate in
open source) as a force multiplier for astronomers to accomplish
more than they ever could before. Using these industry standard tools that
developers know and love, it could also be a boon to recruiting talented
software developers to astronomy.


\bibliography{I6.1}

% if we have space left, we might add a conference photograph here. Leave commented for now.
% \bookpartphoto[width=1.0\textwidth]{foobar.eps}{FooBar Photo (Photo: Any Photographer)}


\end{document}

